\documentclass[12pt]{article}


\begin{document}

\section*{Parameter estimation and uncertainty quantification applied to
advection-diffusion problems arising in atmospheric sources inversion}


\subsection*{abstract}
In this thesis we present a method to obtain an efficient algorithm
to perform parameter estimation with uncertainty quantification 
of mathematical models that are complex and computationally
expensive. We achieve this with a combination of emulation of the mathematical
model using Gaussian processes  and Bayesian statistics and inversion for the parameter estimation
and uncertainty quantification.
In particular we apply these ideas to a source inversion problem in atmospheric dispersion.

We  explain the theory and ideas behind each relevant part 
of the process in the emulation and parameter estimation. The concepts and methodology presented 
in this work are general and can be applied
to a wide range of problems where it is necessary to estimate parameters  but the underlying
mathematical model is expensive, rendering  more classical approaches unfeasible.

To validate the concepts used, we perform a parameter estimation study
in a model that is relatively cheap to compute and whose parameter values are known in 
advance. Finally we perform a parameter estimation with
uncertainty quantification of a much more expensive atmospheric dispersion model
using real data from a lead-zinc smelter  in Trail, British Columbia. 
The parameter estimation includes  approximating high-dimensional integrals
with Markov chain Monte Carlo methods and solving the source inversion problem
in atmospheric dispersion using the Bayesian framework.

\end{document}

