\documentclass[12pt]{article}
\usepackage{cite}
\usepackage{amssymb}
\usepackage{amsthm}
\usepackage{amsmath}
\usepackage[pdftex]{graphicx}
\usepackage{setspace}
\usepackage{mathrsfs}
\usepackage{float}
\usepackage{pbox}
%\usepackage[linesnumbered,ruled]{algorithm2e}
\usepackage{algorithm}
\usepackage{algpseudocode}
\usepackage{setspace}
\setstretch{1.5}

\usepackage{fancyhdr}
\pagestyle{fancy}
\lhead{}
\chead{}
\rhead{}
\lfoot{}
\cfoot{\thepage}
\rfoot{}
%\usepackage{fancyhdr,floatpag}

%\pagestyle{fancy}
%\fancyhf{}% Clear page header/footer
%\renewcommand{\headrulewidth}{0pt}% No header rule
\fancyfoot[C]{\thepage}

%\floatpagestyle{fancy}% Page style for float-page only
%Theorems
\newtheorem{definition}{Definition}




%Commands

%Prior, posterior and likelihood
\newcommand{\post}{\mathbb{P}_{post}}
\newcommand{\like}{\mathbb{P}_{like}}
\newcommand{\prior}{\mathbb{P}_{prior}}
\newcommand{\p}{\mathbb{P}}
\newcommand{\q}{\textbf{q}}
\newcommand{\pars}{p,z_{0},L}
%Other commands
\newcommand{\E}{\mathbb{E}} %Expectation
\newcommand{\tvs}{\mathscr{T}} %TVS symbol
\newcommand{\x}{\textbf{x}}
\newcommand{\y}{\textbf{y}}
\newcommand{\vv}{\textbf{v}}
\newcommand{\argmax}{\mathop{\mathrm{argmax}}}
\newcommand{\dv}{\nabla\cdot}
\DeclareMathOperator*{\cov}{cov}
\DeclareMathOperator{\dom}{dom}
%\DeclareMathOperator{\det}{det}

\begin{document}
\section*{Dispersion Modeling of Air Pollutants in the Atmosphere: a review}
\begin{itemize}
\item Pollutants can be released either by approximated point sources or area sources
\item Transport and transformation of air pollutants in the atmosphere is mainly governed
by advection (wind field). Other processes such as turbulence, chemical reactions and
radioactive dacay can play an important role.
\item lifetime of Zinc is not know, however it is suspected to be days or weeks.
\item The model that we are using is the atmospheric transport equation.
\item Near-surface atmospheric turbulence is caused by mechanical and thermal effects.
Friction forces in a viscuous flow generate mechanical turbulence, which is driven by
wind shear. Therefore \textbf{Three dimensional wind data} is necessary to 
estimate the intensity of mechanical turbulence. However near-surface wind
shear can be estimated using the surface roghness ($z_{0}$). This
parameter represents the strength of the friction between the surface
and the atmosphere.
\item Turbulent mixing in the atmosphere is driven by wind shear and bouyancy.
The approached used to calculate the turbulence parameters is to define
categories with respect to variables such as:
	\begin{itemize}
	\item radiation paramters
	\item wind speed
	\item surface roughness
	\item cloud cover
	\end{itemize}
Here is where the Pasquill stability clases come into play.

\item \textbf{Monin-Obukhov theory}: The M-O length is a reference scale
for boundary layer turbulence, so any height $z$ can be
given instead as $\frac{z}{L}$. Monin-Obukhov theory
is one of the most reaiable basis for all fields of atmospheric modelling

\item There are three ways to model dispersion
	\begin{itemize}
	\item \textbf{Gaussian Dispersion Models}: See derivation in John's paper
		\begin{itemize}
		\item Advantages: Are applied when robust model setup and fast
		response time are key factors.
		\item Disadvantages: Provide poor result with low wind speeds
		where the three dimensional diffusion is significant.
		\end{itemize}
	\item \textbf{Lagrangian models}: Uses the trajectory of each particle and uses ODEs to model
		\begin{itemize}
		\item Advantages: Computational cost is indept. of the output grid. they
		are incredibly efficient for short-range dynamics.

		\item Disadvantages: For long-range simulations the computational cost increase
		significantly.
		\end{itemize}
	\item \textbf{Eulerian models}: The idea is to solve the atmospheric equation in a fixed reference frame
		\begin{itemize}
		\item Advantages: Multipurpose
		\item Disadvantages: Its computational cost
		\end{itemize}
	\end{itemize}
\end{itemize}
\newpage
%%%%%%%%%%%%%%%%%%%%%%%%%%%%%%%%%%%%Paper # 2 %%%%%%%%%%%%%%%%%%%%%%%%%%%%%%%%%%%%%%%%%%%%%%%%%%%%%%%\
\section*{Source Estimation Methods for Atmospheric Dispersion}

\begin{itemize}
\item There are two methods for source estimation. Forward and backward modeling. 
For forward modeling we have topics such as Bayesian updating. Backward modeling
uses only one run backward.

\item Given the meteorology and concentrations observed at several detectors,
these methods characterize the source type and estimate the source strength and location.

\end{itemize}

\newpage
\section*{Uncertainty analysis in Atmospheric Dispersion Modeling}
\begin{itemize}
\item Take into account that in general, observations and model predictions
are uncertain.
\item Once the input parameters to be studied are identified, uncertainty analysis is a two-step sequential procedure.
The first step involves assigning a pdf, The second step involves propagating the joind pdf through the model
\item In the lack of a body of theoretical knowledge it is expected that the judgement of experts
differs considerably to one another. The knowledge of an expert is called \textbf{Substantive expertise}
\end{itemize}



\end{document}
